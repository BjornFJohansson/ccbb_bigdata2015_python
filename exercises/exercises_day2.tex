\documentclass{article}[12pt]
\usepackage[margin=0.75in]{geometry}
\usepackage{color}
\usepackage{graphicx}
\usepackage{hyperref}
\usepackage{fancyvrb}
\usepackage{palatino}
\usepackage{enumitem}
\usepackage[T1]{fontenc} % quotes
\hypersetup{
	colorlinks=true, %set true if you want colored links
	linktoc=all,     %set to all if you want both sections and subsections linked
	linkcolor=blue,  %choose some color if you want links to stand out
	urlcolor=blue
}
\newcommand{\code}[1]{\texttt{#1}}  % let's not use bold for now
\sloppypar
\setlength\parindent{0pt}


\begin{document}

\title{Introduction to Python \\ Day Two Exercises}
\author{Stephanie Spielman \\ \footnotesize{Email: stephanie.spielman@gmail.com}}
\date{}
\maketitle{}

\section{Working with dictionaries}

\begin{enumerate}[itemsep=5ex]
	
	\item Define this dictionary in Python: \code{molecules = \{"DNA":"nucleotides", "protein":"amino acids", "hair":"keratin"\}}, and perform the following tasks:
	\begin{enumerate}[itemsep=2ex]
		\item Create two lists called \code{molecules\_keys} and \code{molecules\_values}, comprised of the keys and values in \code{molecules}, respectively. Use dictionary methods for this task.
		\item Add the key:value pair \code{"ribosomes":"RNA"} to the \code{molecules} dictionary. Print the dictionary to confirm.
		\item Add yet another key:value pair, \code{"ribosomes":"rRNA"}, to the \code{molecules} dictionary, and print out the new dictionary. Understand why the result contains the key:value pairs shown.
	\end{enumerate}
	
	
	\item Congratulations, you've been hired as a zoo-keeper! Now you have to feed the animals. You received these handy Python dictionaries which tells you (a) to which category each animal belongs, and (b) what to feed each category animal category: \\
	
	\code{category = \{"lion": "carnivore", "gazelle": "herbivore", "anteater": "insectivore", "alligator": "homovore", "hedgehog": "insectivore", "cow": "herbivore", "tiger": "carnivore", "orangutan": "frugivore"\}} \\
	
	\code{feed = \{"carnivore": "meat", "herbivore": "grass", "frugivore": "mangos", "homovore": "visitors", "insectivore": "termites"\} } \\
	
	\noindent Copy and paste these dictionaries into a Python script. Use indexing to determine what you should feed the orangutan and print the result.
	

\end{enumerate}

\vspace{2.25cm}

\section{If/elif/else Exercises}

\begin{enumerate}[itemsep=5ex]
	\item In Texas, you can be a member of the elite "top 1\%" if you make at least \$423,000 per year. Alternatively, in Hawaii, you can be a member once you start making at least \$279,000 per year! Finally, if you live in New York, you need to earn at least \$506,000 a year to make the cut. \\\\ Andrew is CEO of Big Money Company, and he earns \$425,000 per year, and Stacey is CEO of Gigantic Money Company with an annual salary of \$700,000. Use if-statements to determine, and print, whether Andrew and Stacey would be considered top 1\%-ers in Texas, Hawaii, and New York.
	
	\item Copy and paste the list below into a Python script and perform the following tasks. \\ \code{b = [19, 3, 2, 88, 82, 31, -9, 8, 33, -6, -111]} 
	\begin{enumerate}[itemsep=2ex]
		\item Write an if/else statement to determine whether the length of list \code{b} is above or below 10. If this condition is true, create a new list called \code{newb} that contains the first 5 numbers in b. If this condition is false, create a new list called \code{newb} that contains the \emph{last 5 numbers} from list \code{b}. 
		\item Determine the sum of the list \code{newb} (Hint: use the \code{sum()} function).
		\item Use an if/else statement to determine if this sum is even or odd (Hint: use the modulus operator, \code{\%}), and print a sentence summarizing the result.
	\end{enumerate}

	
		   
\end{enumerate}
			
\vspace{2.25cm}

\section{For-loop Exercises}


\begin{enumerate}[itemsep=5ex]
	\item Write a for-loop to print the powers of 2, ranging from 2$^0$ to 2$^{15}$. (Note that in Python, the exponent symbol is **, as in 3**2 = 9). 
	
	\item Modify your code from (1) such that all each power-of-2 value is saved to a list called \code{powers2}. Print the resulting list. 

%\item Modify your code from (1) to print the exponents, ranging 0 - 15, for all numbers in the range 2 - 10. Each printed line should read "The value of x raised to y is z.", where x, y, and z have been replaced with the appropriate values. 
	
	\item This is the longest German word\footnote{Sadly, this \href{http://www.salon.com/2013/06/03/law_change_spells_end_for_germanys_longest_word_ap/}{AP article} reported in 2013 that this word has been stricken from the dictionary.}: "rindfleischetikettierungsuberwachungsaufgabenubertragungsgesetz" (accents removed), meaning "widow of a Danube steamboat company captain". Save this word to a string variable in Python, and count the number of times each letter in the alphabet (a-z) appears in this word. For each letter, print the number of occurrences, e.g.\ "a: 4". 
	\\ Hints:
	\begin{itemize}
		\item Use a \emph{nested for-loop} structure. The outer loop should loop over the word, and the inner loop over the alphabet.		\item You can loop over strings just like lists.
	\end{itemize} 
	Now, modify this code in several ways:
	\begin{enumerate}[itemsep=2ex]
		\item Modify your code such that letters with a count of 0 do not get printed. 
		\item Modify your code from the previous point such that it uses the \code{in} operator. This operator returns True or False, as in: \code{"a" in "apple"} returns True, and \code{3 in [4,5,6,7,8]} returns False.
		\item Modify your code to save the letter counts as you loop. For this, you will need to define a variable to store the sum, and you can increment this variable with counts as you go. Once this is complete, use an if-else statement to check that the total sum is equal to length of the full word. Print a statement indicating whether your sum is correct or not.
	\end{enumerate}
	
	
	\item A silly professor has decided to curve grades in a very special way: grades above 95 are reduced by 10\%, grades between 75-95 (inclusive) remain the same, and grades below 80 are raised by 10\%. You have been tasked with crunching the numbers.
	
	\begin{enumerate}[itemsep=2ex]
		\item Create a list of new grades that reflects these rules from the following grade list: \\
		\code{grades = [45, 94, 25, 68, 88, 95, 72, 79, 91, 82, 53, 66, 58]} 
		 
		\item The \emph{nested} list below contains three sets of grades for silly professor's three classes: \\ 
		\code{all\_grades = [[45, 94, 25, 68, 88, 95, 72, 79, 91, 82, 53, 66, 58], [23, 46, 17, 67, 55, 42, 31, 73], [91, 83, 79, 76, 82, 91, 95, 77, 82, 77]]}
		\\ Create a new nested list with the curved grades for each of these classes.
	\end{enumerate}
	
	\item Using the zoo-keeper dictionaries from today's second dictionary exercise, loop over the animals in \code{category} and, using the \code{feed} dictionary, determine and print what food each animal should eat, e.g.\ "The gazelle should eat grass." 

	\item This dictionary provides the molecular weight for all amino acids: \\ \code{ amino\_weights = \{'A':89.09, 'R':174.20, 'N':132.12, 'D':133.10, 'C':121.15, 'Q':146.15, 'E':147.13, 'G':75.07, 'H':155.16, 'I':131.17, 'L':131.17, 'K':146.19, 'M':149.21, 'F':165.19, 'P':115.13 'S':105.09, 'T':119.12, 'W':204.23, 'Y':181.19, 'V':117.15\}}. \\ Using this dictionary, determine the molecular weight for this protein sequence: \\ \code{"GAHYADPPLVKMPWWRTHHHC"}.
	
	\item The strings methods \code{.beginswith()} and \code{.endswith()} are useful for determining if a given string begins/ends with a particular substring, and they return True or False. For example, \code{"banana".beginswith("ba")} returns True, and \code{"oranges".endswith("W")} returns False. Note that both methods are \emph{case-sensitive}, meaning that upper- and lower-case matter! \\\\ Using the method \code{.beginswith()}, determine if the sentence "Dan's dog, dubbed Fluffy, dove deep in the dam and drank dirty water, but he didn't drown" is an alliteration for the letter "d". For the purposes of this example, let's assume that if at least 50\% of the words start with the same letter, then it is alliteration. Otherwise, it is not. 
	\\ Hint: the methods \code{.split()} and \code{.lower()}, as well as the function \code{len()} will be useful.


\end{enumerate}



\end{document}
