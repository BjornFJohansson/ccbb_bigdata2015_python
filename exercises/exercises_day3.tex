\documentclass{article}[12pt]
\usepackage[margin=0.75in]{geometry}
\usepackage{color}
\usepackage{graphicx}
%\usepackage{hyperref}
\usepackage{fancyvrb}
\usepackage{palatino}
\usepackage{enumitem}
\usepackage{natbib}
\usepackage[T1]{fontenc} % quotes
%\hypersetup{
%	colorlinks=true, %set true if you want colored links
%	linktoc=all,     %set to all if you want both sections and subsections linked
%	linkcolor=blue,  %choose some color if you want links to stand out
%	urlcolor=blue
%}
\newcommand{\code}[1]{\texttt{#1}}  % let's not use bold for now
\sloppypar
\setlength\parindent{0pt}
\usepackage{textcomp}

\begin{document}

\title{Introduction to Python \\ Day Three Exercises}
\author{Stephanie Spielman \\ \footnotesize{Email: stephanie.spielman@gmail.com}}
\date{}
\maketitle{}

\section{Defining Functions}

For this set of exercises, you will re-write some of yesterday\textquotesingle s exercises as functions. The code doing the actual computation will remain virtually the same, except it will be written in the context of a function and subsequently called. \textbf{ After you write each function, run it with 2-3 test cases to confirm that it works as expected!!}

\begin{enumerate}[itemsep=5ex]

	\item In Texas, you can be a member of the elite "top 1\%" if you make at least \$423,000 per year. Alternatively, in Hawaii, you can be a member once you start making at least \$279,000 per year! Finally, if you live in New York, you need to earn at least \$506,000 a year to make the cut. \\\\ Write a function to determine if a given salary is a "top-1\%" salary in Texas, Hawaii, and/or New York. Your function should take a single argument, the salary, and \emph{print} a sentence indicating in which state(s) this salary is and is not a top-1\% salary. Your function should not return a value.
	
	
	\item Write a function that returns a list of the powers (exponents 0-15) for a provided number. This function should take 1 argument: the number to raise to powers 0-15.
	
	\item Write a function to curve a list of grades, silly-professor style. This function should take \emph{four arguments}:
	\begin{itemize}
		\item A list of grades to curve
		\item The cutoff \emph{above which} grades are reduced
		\item The cutoff \emph{below which} grades are raised
		\item The scaling value
	\end{itemize}
	The function should return a list of curved grades.
	
	\item Write a function to compute the molecular weight of a protein sequence. This function should take a single argument, a protein-sequence string, and it should return a single value, the molecular weight. Your function should account for the potential presence of ambiguous amino acids (again, compute these weights from average of all weights). Once your function is written, run it on the protein sequence "ABVPOXIRBTQQWS." \\ Use this dictionary in your function:
	\\ \code{ amino\_weights = \{"A":89.09, "R":174.20, "N":132.12, "D":133.10, "C":121.15, "Q":146.15, "E":147.13, "G":75.07, "H":155.16, "I":131.17, "L":131.17, "K":146.19, "M":149.21, "F":165.19, "P":115.13, "S":105.09, "T":119.12, "W":204.23, "Y":181.19, "V":117.15\}} 
	
	\item Write a function to determine if a sentence is an alliteration. Your function should take two arguments: the sentence and the alliteration letter. Your function should return the value \code{True} if the sentence is alliteration and \code{False} otherwise. Importantly, note that you can us an \code{if} statement to return different values under different conditions!
		
\end{enumerate}




\bibliographystyle{plainnat}
\bibliography{citations}
















\end{document}
