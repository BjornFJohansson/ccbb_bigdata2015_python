\documentclass{article}[12pt]
\usepackage[margin=0.75in]{geometry}
\usepackage{color}
\usepackage{graphicx}
%\usepackage{hyperref}
\usepackage{fancyvrb}
\usepackage{palatino}
\usepackage{enumitem}
\usepackage{natbib}
\usepackage[T1]{fontenc} % quotes
%\hypersetup{
%	colorlinks=true, %set true if you want colored links
%	linktoc=all,     %set to all if you want both sections and subsections linked
%	linkcolor=blue,  %choose some color if you want links to stand out
%	urlcolor=blue
%}
\newcommand{\code}[1]{\texttt{#1}}  % let's not use bold for now
\sloppypar
\setlength\parindent{0pt}


\begin{document}

\title{Introduction to Python \\ Day Four Exercises}
\author{Stephanie Spielman \\ \footnotesize{Email: stephanie.spielman@gmail.com}}
\date{}
\maketitle{}

\section*{Working Sequence Data in BioPython}


\begin{enumerate}[itemsep=5ex]

	\item The first exercise will focus on manipulating sequences in the file "dopamine\_sequences.fasta". Perform the following tasks:
	
	\begin{enumerate}[itemsep=2ex]
		\item Use the SeqIO module to read and save records from the file "dopamine\_sequences.fasta".
		\item Determine the length for each sequence, and save this information to a dictionary. The keys should be the sequence ID, and the values should be the length.
		\item Translate each sequence to amino acids, and count the number of tryptophans (coded as "W") per sequence. Again, save this information in a dictionary. (Hint: BioPython has implemented certain useful string methods, like \code{.count()} for \code{Seq} object sequences.)
		\item Re-do parts (b) and (c) by writing \emph{functions} to count the sequence length and determine the number of tryptophans. Each function should take a \emph{single} \code{SeqRecord} object as the argument.
		
		
	\end{enumerate}
	
	\item The first exercise will focus on manipulating sequences in the sequence alignment file "dopamine\_alignment.fasta", from \citet{SpielmanKumarWilke2015}. This file contains the same sequences as does "dopamine\_sequences.fasta", except the sequences are aligned and in amino-acid space. Perform the following tasks:
		
	\begin{enumerate}[itemsep=2ex]
		\item Use the SeqIO module to read and save records from the file "dopamine\_alignment.fasta".
		\item Translate each records to amino-acids. These sequences are known to have a motif "NPxxY", where the "xx" pair is either "II", "VV", or "VI", towards the end of their sequence. In this alignment, this motif appears at positions 745 - 750 (although remember that Python indexing starts at 0!). Determine how many sequences have each motif, and then print your results to screen (e.g. \# sequences have NPIIY, \# sequences have NPVIY, and \# sequences have NPVVY). Make sure that the total number of motifs you count equals the number of sequences in the file (you should code this, not manually/visually check it).
		\item Save these sequences to a new file, called "dopamine\_alignment.phy" in phylip format. Perform this step twice, once using the \code{.convert()} method and once using the \code{.write()} method.
			
		\end{enumerate}
	
\end{enumerate}	
	





\bibliographystyle{plainnat}
\bibliography{citations}
















\end{document}
