\documentclass{article}[12pt]
\usepackage[margin=0.75in]{geometry}
\usepackage{color}
\usepackage{graphicx}
%\usepackage{hyperref}
\usepackage{fancyvrb}
\usepackage{palatino}
\usepackage{enumitem}
\usepackage{natbib}
\usepackage[T1]{fontenc} % quotes
%\hypersetup{
%	colorlinks=true, %set true if you want colored links
%	linktoc=all,     %set to all if you want both sections and subsections linked
%	linkcolor=blue,  %choose some color if you want links to stand out
%	urlcolor=blue
%}
\newcommand{\code}[1]{\texttt{#1}}  % let's not use bold for now
\sloppypar
\setlength\parindent{0pt}
\usepackage{textcomp}

\begin{document}

\title{Introduction to Python \\ Day Four Exercises}
\author{Stephanie Spielman \\ \footnotesize{Email: stephanie.spielman@gmail.com}}
\date{}
\maketitle{}


\section{Interacting with files}

These questions make use of a series of files included in this zip directory. 

\begin{enumerate}[itemsep=5ex]
	
	\item Open the file "flu\_sequences.fasta" in read-mode, and save the contents of a file as a single string using the \code{.read()} method (be sure to close the file, or use \code{with} control-flow!). Address the following questions using Python:
	
	\begin{enumerate}[itemsep=2ex]
		\item How many \emph{characters} are in this file? Remember that \code{.read()} saves the entire file as a single string!
		\item How many \emph{lines} are in this file? (Hint: use the \code{.strip()} method as part of the solution).
		\item Use a for-loop to print the first 10 lines of the file.
		\item Open and read the file again, except this time, use the \code{.readlines()} method instead of \code{.read()}. Perform the same three tasks. (Hint: to count the characters, you can either loop over the line list, or use the string method \code{.join()}. This method is the "opposite" of \code{.split()} -- it joins together a list of strings (argument) using a given separator, e.g. \code{" ".join(["a", "b", "c", "d"])} returns the string \code{"a b c d"}.)
	\end{enumerate}
	
	\item Write a function to return a list of lines in a given file. This function should take one argument, the file name, and return a list of the lines. Use this function as part of your code to perform these tasks on the included files named file1.txt, file2.txt, file3.txt,..., file20.txt (below, they are referred to as file<1-20.txt> for convenience). (Note that the code which generated these files is included in the python scipt \code{generate\_files.py}.)
	\begin{enumerate}[itemsep=2ex]
		\item Determine the number of lines in each file. Use a for-loop to loop over files. When writing this for-loop, think about the cleanest way to loop over these file names (hint: it is \emph{not} to loop over a list with contents \code{["file1.txt", "file2.txt", ...]}). Print, for each file, a sentence that reads "file1.txt contains XXX" lines (with file1.txt and XXX replaced accordingly).
		\item For each file<1-20>.txt, create new files called "file<1-20>\_upper.txt" which contain the full contents of each file, except entirely in upper-case. Note that this is a 2-step process for each file: first read in, and then write the new file. 
	\end{enumerate}
	
	\item Create a new file called "file\_full.txt" which contains all contents from each file<1-20>.txt (in order). Perform this task twice using two different approaches:
	
	\begin{enumerate}[itemsep=2ex]
		\item Open the "file\_full.txt" a single time for writing and write all contents to this handle.
		\item Using the "append" mode to open and close the file multiple times for writing.
	\end{enumerate}
	Hint: think about whether the \code{.read()} or \code{.readlines()} method makes the most sense for this task.
	
	\item For this question, you are going to save some useful information to a file. Specifically, you will create a \emph{CSV} (comma-separated values) file. This is basically what Excel files are - files with columns of data. The top row is the header, and each subsequent row contains data. Columns of data are separated by commas in csv files (note that there are other types of \emph{delimited} files, like tab-delimited files). 
	\\\\ You will use the data from the file "dopamine\_sequences.fasta," which contains DNA sequences of vertebrate dopamine receptors published in \citet{SpielmanKumarWilke2015}. For convenience, the separate file "dopamine\_sequences\_dictionary.txt" contains a Python dictionary of these sequences; the keys are sequences IDs, and the values are the sequences. You can copy and paste the contents of "dopamine\_sequences\_dictionary.txt" into the Python script you\textquotesingle ll write for this question.
	\\\\ Now, create a CSV file with two columns: the sequence ID and the sequence length. Follow this general strategy:
	\begin{enumerate}[itemsep=2ex]
		\item Collect all information to write to the output file (this has been done for you in the dictionary!)
		\item Open the output file and write a header to the file (remember to add \code{"\textbackslash n"} for newlines!)
		\item Write each line to the output file. In this case, you will need to loop over the dictionary and write each key:value pair in CSV format with a newline character, e.g. "sequenceID,length\textbackslash n"
	\end{enumerate}
	Once you have written the file, open it in a text editor to be sure that it is correct.
	
	Now, modify this code in the following ways:
	\begin{enumerate}[itemsep=2ex]
		
		\item Modify your CSV-writing code such that it uses a function to write the CSV. This function should take three arguments: the header string, the dictionary to write, and the output file name. The function does not need to return anything.
		
		\item Modify your code and function so that you now write a file with three columns: sequence ID, sequence length, and percent of ambiguous nucleotides per DNA sequence. In this sequence, the character "N" is considered ambiguous (unlike A, C, G, T).
		
	\end{enumerate}
\end{enumerate}	

	


\bibliographystyle{plainnat}
\bibliography{citations}
















\end{document}
